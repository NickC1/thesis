\newpage


\section{CONCLUSION}

Collection of remote images of the nearshore region and photomosaics of the coral reef benthos has provided a series of spatiotemporal images of the upper shore-face and a series of spatial images along a gradient of coral reef degradation. Forecast skill from the novel application of both spatiotemporal nonlinear time series forecasting and the spatial analog suggest that the evolution of coastal and coral zones are governed by internal non-linear dynamics. Both forecast techniques elucidated the relative amount of stochasticity in the respective systems.

Although the metric was able to determine the relative amount of stochasticity within the systems, it does not provide an absolute measure of determinism across systems generated by different dynamics. For example, the absolute value of the deterministic metric was not able to capture the amount of determinism in the one-dimensional periodic equation when compared to the one-dimensional chaotic map. Moreover, the technique does not provide insight into the actual dynamics that generate a system. For the coral images, nothing is discovered concerning what spatial dynamics created the benthos other than the fact that the final configuration is deterministic.

The largest drawback of the technique is the large amount of data needed while the system is in it's steady state. Incomplete reconstructed state spaces result in forecast skill that mimics noise within the system even if the system is perfectly deterministic. Experimentation with small amounts of data for the one-dimensional chaotic map have shown suppressed $R^2$ values even for perfectly deterministic systems. Additionally, for high dimensional reconstructed state spaces, it is difficult to tell whether the system is in not in a steady state or the system is suffering from an incomplete attractor. This was apparent with the post-nourishment beach. The system most likely was not in a steady state, but it is also possible that the phase space for the post-nourishment beach is incomplete and more data is needed.

Nonetheless, the technique presented here was able to clearly distinguish the different system evolution for the pre-nourishment beach and post-nourishment as well as the different spatial configurations along the gradient of coral degradation. Likewise, the deterministic metric was able to succinctly illuminate the non-linear influence between near neighbors in the reconstructed state space between both the mathematical equations along a gradient of noise and natural systems (beach and coral). With significant human enterprise located in coastal regions around the world and the high economic benefit of coral reefs, there is a premium placed on the ability to understand coastline behavior and coral reef health. The results presented here suggest that image data coupled with nonlinear forecasting techniques can result in an insight into the evolution of the systems in time or space.

% **NEED TO LIST DRAWBACKS AND CAVEATS in a paragraph**
% 1) The metric is only good for comparing within system and does not provide an absolute measure of randomness vs determinism.

% 2) Embedding dim and lag values can not be set independently of the analysis.  Have attempted to set those values in a manner consistent with other authors.

% 3) The technique does not provide insight on what the dynamics actually are.

% 4) The technique requires a large amount of data and it requires data taken while the system is within it's attractor.  As shown for the beach, this can cause problems and it's often difficulty to identify whether a system is in the attractor.

% The methods used here do require choices for parameter values but some values, such as lags in the nonlinear forecasting technique, can be determined according to explicit tests that are not based on data fitting.









