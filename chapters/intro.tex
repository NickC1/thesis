\newpage
\pagenumbering{arabic}
\thispagestyle{empty}


\section{INTRODUCTION}
\thispagestyle{empty}


Emergence of large-scale coherence and patterning in systems of many interacting constituents is a hallmark of complex systems, in which feedbacks and nonlinear internal dynamics dominate evolution \cite{emergence}. The extent to which evolution of the system is controlled by internal nonlinear dynamics, as opposed to responding primarily to the noisy forcing environment, is difficult to quantify. Most numerical models that simulate system dynamics rely heavily on detailed forcing information to make forecasts and often only account for weakly nonlinear intrinsic dynamics. In contrast, the methods presented here attempt to forecast based solely on previous states in both space and time without direct knowledge of concurrent forcing or model equations of system dynamics. The efficacy of these predictions does, however, provide insight about the dynamics that dominate the evolution of the system.

Techniques that model the evolution of macroscopic features, like shoreline position or species distribution, using process-based approaches that ramp up granular physics or detailed chemical and physical species level dynamics via explicit parameterizations are well established \cite{beach_model_physics} \cite{coral_model}. However, these techniques have suboptimal forecast performance and site adaptability. Internal nonlinear dynamics have also been shown to influence a system's response to forcing, where behavior may not simply be connected to forcing signatures in time. For example, sediment transport systems can exhibit behavior that masks supply signals within certain ranges of amplitude, outside of which the forcing appears to overbear these dynamics\cite{shredding_signals}. These examples and others illustrate the importance of nonlinear internal dynamics in contributing to the evolution of natural phenomena without a need for knowledge of fast-time scale dynamics or detailed forcing features.

The relative position of the ocean coastline is an important resource, as it dictates the usable recreational beach and influences property values \cite{economic}. Despite its importance to static coastal communities, the region is highly dynamic, varying with tidal elevation, wave set-up, and storm-surge. The magnitude of change in coastline position is regulated by the local slope of the shore-face, the profile of which is not typically linear \cite{beach_profiles}. Characteristics (shape) of the intertidal beach and surf-zone profile are known to adjust in response to changing environmental conditions, and adjustments are not necessarily uniform along the profile \cite{equilibrium_profiles} \cite{equilibrium_profiles2}. Interestingly, despite the myriad of hydrodynamic forcings and sediment compositions, sandy coastlines exhibit a limited range of morphological modes.

Likewise, coral reefs provide valuable economic value to their adjacent communities through tourism and food provision \cite{coral_economics}. These valuable coral reefs, however, are susceptible to drastic changes due to human induced activities such as fishing, pollution, and climate change\cite{coral_degredation}.  In the most extreme examples of these disturbances, coral reefs can systematically transition to a benthos covered entirely in algae species \cite{coral_degredation}.  Similar to coastlines then, despite a myriad of complicated chemical, biological, and physical interactions at short time scales, coral reefs appear to exhibit a small range of benthic configurations, either healthy with coral species covering the benthos or disturbed with algae out-competing all species for space\cite{phase_shifts}. 

As a study domain, with the advent of remote imaging systems and new underwater photographing techniques, the beach and coral environments are no longer data poor\cite{beach_imagery}. Comprehensive observations and the development of techniques to extract rigorous, quantifiable features from these sources have led to the development of data driven modeling techniques and forecasting\cite{beach_predicting}. Time series obtained from imaging systems contain information about internal dynamics as they manifest in the system’s evolution\cite{beach_attractor}. According to Takens’ Theorem, given sufficient data, a deterministic system’s phase space trajectory is reproducible and system evolution may be forecast\cite{original_rubin}\cite{chaotic_analysis}. Specifically, forecasting is based on neighbor trajectories within the embedded phase space, and skill can be affected by the choice of embedding dimension, weighting of neighbor trajectories, the amount of noise in the data, and the prevalence of nonlinear interactions\cite{Embedology}\cite{measurement_error_vs_chaos}.

These nonlinear forecasting techniques have proven capable of outperforming mechanistic models in noisy, non-linear ecological systems, and they have shown the ability to distinguish noisy natural time series from those governed by nonlinear dynamics \cite{natural_series_classification} \cite{rubin_lagged_vals}. 

Here, techniques in nonlinear time series analysis and forecasting are used to explore the extent to which local nonlinear interactions affect day-to-day intertidal profile adjustments and species distribution on coral reefs. Additionally, a metric is created to quantify the relative role of deterministic non-linear interactions and random behavior.   
